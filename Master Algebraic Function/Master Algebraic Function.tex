\documentclass[12pt, letterpaper, twosided]{article}
\usepackage[utf8]{inputenc}
\usepackage[english]{babel}
\usepackage{multicol}
\setlength{\columnsep}{1cm}
\usepackage[legalpaper, landscape, margin=2in]{geometry}
\usepackage{amsfonts}
\usepackage{amsmath}
\usepackage{amssymb}
\usepackage{titlesec}
\usepackage{ulem}

%  https://www.overleaf.com/learn/latex/Page_size_and_margins

\title{Master Algebraic Function\footnote{One formula to rule them all.}}
\author{\textit{C.M. Hurley}}
\date{Apirl 20th, 2020}

\titleformat
{\chapter}
[display]
{\bfseries\Large\itshape}
{Section \ \thechapter}
{0.5ex}
{\centering}
[\vspace{-0.5ex}]

\pagenumbering{roman}


	\begin{document}
	\maketitle
	\newpage % Puts contents on new page...
	\tableofcontents

\newpage
\section*{Preface}%PREFACE%
\addcontentsline{toc}{section}{Preface/Abstract}
\begin{multicols}{3}
[
All things unknown are found by subtracting by all relevant knowns or by happen chance; the latter is the cause of this paper.
]
\hspace{0.5cm}If a set of values isn't constant after taking the rate\footnote{Before moving forward, the \textit{Rate} is the total amount of columns of differences taken. And the last column, consisting of a list of constants or a single number, is said to be the \textit{Rate of Difference}(RoD) for a set of digits; see the paper on Rate of Difference for more information.} as many times as possible, or simply put, doesn't have a RoD then it  is therefore not of an algebraic function. But it \textit{is} possible to put just those numbers into what's dubbed as a \textit{Partial Function} which uses the same method as if it were algebraic, but differs in the fact that when the last subtraction is performed to obtain the RoD, the difference is taken as a sort-of partial RoD. Streamlined in the following statement, "If this last value were to remain constant, then the set of numbers used to get the RoD are partly algebraic." The catch is that the greater the set of values, therefore a higher order algebraic function is needed to represent them; on account of a much greater Rate needed to be taken. From this one could only imagine the near infinite coefficients used to skillfully evaluate and produce a function for a huge list of obscure values. But don't fret because the efforts in obtaining these \textit{Partialities} could be exploited and analyzed to better determine a real function for the entire set of numbers in question; think of it as a partial solution.
\newline Partial Functions have been successfully applied to primes and combinatorics and the extent of those solutions is provided in the end.
\end{multicols}


\pagenumbering{arabic}

\newpage
\section{Rate of Difference}%MAIN FORMULA%
\setcounter{page}{1}
\hspace{0.5cm}short summary of idea...


	\begin{multicols}{3}
[
	\subsection{Deriviation}
All human things are subject to decay. And when fate summons, Monarchs must obey.
]
\hspace{0.5cm}Hello, here is some text without a meaning.  This text should show what 
a printed text will look like at this place.
If you read this text, you will get no information.  Really?  Is there 
no information?  Is there...
	\end{multicols}


	\begin{multicols}{3}
[
	\subsection{Algebraic Combinatoric Forms}
All human things are subject to decay. And when fate summons, Monarchs must obey.
]
\hspace{0.5cm}Hello, here is some text without a meaning.  This text should show what 
a printed text will look like at this place.
If you read this text, you will get no information.  Really?  Is there 
no information?  Is there...
	\end{multicols}



\newpage
\section{Partial Functions}%NON-ALGEBRAICS%
\hspace{0.5cm}short summary of idea...

	\begin{multicols}{3}
[
	\subsection{test}
All human things are subject to decay. And when fate summons, Monarchs must obey.
]
\hspace{0.5cm}Hello, here is some text without a meaning.  This text should show what 
a printed text will look like at this place.
If you read this text, you will get no information.  Really?  Is there 
no information?  Is there...
	\end{multicols}



\newpage
\section*{End}%FINAL THOUGHTS%
\addcontentsline{toc}{section}{End}
\hspace{0.5cm}test




	\end{document}